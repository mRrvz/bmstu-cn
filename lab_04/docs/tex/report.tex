\documentclass[12pt]{report}
\usepackage[utf8]{inputenc}
\usepackage[russian]{babel}
%\usepackage[14pt]{extsizes}
\usepackage{listings}
\usepackage{graphicx}
\usepackage{amsmath,amsfonts,amssymb,amsthm,mathtools} 
\usepackage{pgfplots}
\usepackage{filecontents}
\usepackage{float}
\usepackage{comment}
\usepackage{indentfirst}
\usepackage{eucal}
\usepackage{enumitem}
%s\documentclass[openany]{book}
\frenchspacing

\usepackage{indentfirst} % Красная строка

\usetikzlibrary{datavisualization}
\usetikzlibrary{datavisualization.formats.functions}

\usepackage{amsmath}


% Для листинга кода:
\lstset{ %
	language=c,                 % выбор языка для подсветки (здесь это С)
	basicstyle=\small\sffamily, % размер и начертание шрифта для подсветки кода
	numbers=left,               % где поставить нумерацию строк (слева\справа)
	numberstyle=\tiny,           % размер шрифта для номеров строк
	stepnumber=1,                   % размер шага между двумя номерами строк
	numbersep=5pt,                % как далеко отстоят номера строк от подсвечиваемого кода
	showspaces=false,            % показывать или нет пробелы специальными отступами
	showstringspaces=false,      % показывать или нет пробелы в строках
	showtabs=false,             % показывать или нет табуляцию в строках
	frame=single,              % рисовать рамку вокруг кода
	tabsize=2,                 % размер табуляции по умолчанию равен 2 пробелам
	captionpos=t,              % позиция заголовка вверху [t] или внизу [b] 
	breaklines=true,           % автоматически переносить строки (да\нет)
	breakatwhitespace=false, % переносить строки только если есть пробел
	escapeinside={\#*}{*)}   % если нужно добавить комментарии в коде
}


\usepackage[left=2cm,right=2cm, top=2cm,bottom=2cm,bindingoffset=0cm]{geometry}
% Для измененных титулов глав:
\usepackage{titlesec, blindtext, color} % подключаем нужные пакеты
\definecolor{gray75}{gray}{0.75} % определяем цвет
\newcommand{\hsp}{\hspace{20pt}} % длина линии в 20pt
% titleformat определяет стиль
\titleformat{\chapter}[hang]{\Huge\bfseries}{\thechapter\hsp\textcolor{gray75}{|}\hsp}{0pt}{\Huge\bfseries}


% plot
\usepackage{pgfplots}
\usepackage{filecontents}
\usetikzlibrary{datavisualization}
\usetikzlibrary{datavisualization.formats.functions}

\begin{document}
	%\def\chaptername{} % убирает "Глава"
	\thispagestyle{empty}
	\begin{titlepage}
		\noindent \begin{minipage}{0.15\textwidth}
			\includegraphics[width=\linewidth]{img/b_logo}
		\end{minipage}
		\noindent\begin{minipage}{0.9\textwidth}\centering
			\textbf{Министерство науки и высшего образования Российской Федерации}\\
			\textbf{Федеральное государственное бюджетное образовательное учреждение высшего образования}\\
			\textbf{~~~«Московский государственный технический университет имени Н.Э.~Баумана}\\
			\textbf{(национальный исследовательский университет)»}\\
			\textbf{(МГТУ им. Н.Э.~Баумана)}
		\end{minipage}
		
		\noindent\rule{18cm}{3pt}
		\newline\newline
		\noindent ФАКУЛЬТЕТ $\underline{\text{«Информатика и системы управления»}}$ \newline\newline
		\noindent КАФЕДРА $\underline{\text{«Программное обеспечение ЭВМ и информационные технологии»}}$\newline\newline\newline\newline\newline
		
		\begin{center}
			\noindent\begin{minipage}{1.1\textwidth}\centering
				\Large\textbf{  Отчет по лабораторной работе №4}\newline
				\textbf{по дисциплине <<Компьютерные сети>>}\newline\newline\newline
			\end{minipage}
		\end{center}
		
		\noindent\textbf{Тема} $\underline{\text{Настройка сетевых служб: DNS, HTTP, электронной почты в сетевом эмуляторе}}$\newline\newline
		\noindent\textbf{Студент} $\underline{\text{Романов А.В.~~~~~~~~~~~}}$\newline\newline
		\noindent\textbf{Группа} $\underline{\text{ИУ7-73Б~~~~~~~~~~~~~~~~~~~}}$\newline\newline
		\noindent\textbf{Преподаватель} $\underline{\text{Рогозин Н. О.}}$\newline\newline\newline
		
		\begin{center}
			\vfill
			Москва~---~\the\year
			~г.
		\end{center}
	\end{titlepage}


\section*{Задание}

\textbf{Вариант №12}

\begin{itemize}
	\item присвоить портам устройств статическим IPv4 адреса в соответсвтвии с вариантом;
	\item настроить безопасный доступ к коммутаторам и маршрутизатору;
	\item указать адреса портов маршрутизатора как ядреа шлюза по умолчанию для конечных узлов;
	\item настроить DNS сервер;
	\item указать адрес DNS сервера для конечных узлов;
	\item настроить почтовый сервер SMTP и POP3;
	\item добавить почтовый клиент на всех ПК;
	\item настроить HTTP сервер, разместить там тестовую страницу с номером варианта, фамилией, номером группы, датой выполнения работы;
	\item проверить корректное прохождение сигнала между всеми узлами сети, доступность настроенных сервисов со стороны клиентов на ПК;
	\item отметить широковещательные домены и домены коллизий на схеме.
\end{itemize}

\section*{Результаты работы}

\subsection*{Присвоить портам устройств статические IPv4 адреса в соответствии с вариантом}

\begin{figure}[H]
	\begin{center}
		\includegraphics[scale=0.4]{img/1.png}
	\end{center}
	\caption{Настройка статического IPv4 адреса конечного узла}
	\label{fig:1}
\end{figure}

Для двух других углов процесс аналогичен.

\subsection*{Настроить безопасный доступ к коммутаторам и маршрутизатору}

Настройка проведена на примере коммутатора.

\begin{figure}[H]
	\begin{center}
		\includegraphics[scale=0.7]{img/2.png}
	\end{center}
	\caption{Настройка безопасного доступа к коммутатору}
	\label{fig:2}
\end{figure}

Для двух других коммутаторов и маршрутизатора процесс аналогичен.

\subsection*{Указать адреса портов маршрутизатора как адрес шлюза по умолчанию для конечных узлов. Указать адрес DNS сервера для конечных узлов}

\begin{figure}[H]
	\begin{center}
		\includegraphics[scale=0.4]{img/3.png}
	\end{center}
	\caption{Указание адреса порта маршрутизатора для сети с ПК}
	\label{fig:3}
\end{figure}

\begin{figure}[H]
	\begin{center}
		\includegraphics[scale=0.4]{img/4.png}
	\end{center}
	\caption{Указание адреса порта маршрутизатора для сети с DNS-сервером}
	\label{fig:4}
\end{figure}

\begin{figure}[H]
	\begin{center}
		\includegraphics[scale=0.4]{img/5.png}
	\end{center}
	\caption{Указание адреса порта маршрутизатора для сети с HTTP и SMTP-серверами}
	\label{fig:5}
\end{figure}

\begin{figure}[H]
	\begin{center}
		\includegraphics[scale=0.4]{img/6.png}
	\end{center}
	\caption{Указание адреса порта маршрутизатора как адреса шлюза по умолчанию и адреса DNS-сервера для ПК}
	\label{fig:6}
\end{figure}

\begin{figure}[H]
	\begin{center}
		\includegraphics[scale=0.4]{img/7.png}
	\end{center}
	\caption{Указание адреса порта маршрутизатора как адреса шлюза по умолчанию и адреса DNS-сервера для DNS-сервера}
	\label{fig:7}
\end{figure}

\begin{figure}[H]
	\begin{center}
		\includegraphics[scale=0.4]{img/8.png}
	\end{center}
	\caption{Указание адреса порта маршрутизатора как адреса шлюза по умолчанию и адреса DNS-сервера для HTTP-сервера}
	\label{fig:8}
\end{figure}

\begin{figure}[H]
	\begin{center}
		\includegraphics[scale=0.4]{img/9.png}
	\end{center}
	\caption{Указание адреса порта маршрутизатора как адреса шлюза по умолчанию и адреса DNS-сервера для SMTP-сервера}
	\label{fig:9}
\end{figure}

\subsection*{Настроить DNS сервер. Добавить почтовые записи на DNS-сервер}

\begin{figure}[H]
	\begin{center}
		\includegraphics[scale=0.4]{img/10.png}
	\end{center}
	\caption{Настройка DNS-сервера}
	\label{fig:10}
\end{figure}

\subsection*{Настроить почтовый сервер SMTP и POP3}

\begin{figure}[H]
	\begin{center}
		\includegraphics[scale=0.4]{img/11.png}
	\end{center}
	\caption{Настройка почтового сервера}
	\label{fig:11}
\end{figure}

\subsection*{Настроить почтовый клиент на всех ПК}

\begin{figure}[H]
	\begin{center}
		\includegraphics[scale=0.4]{img/12.png}
	\end{center}
	\caption{Настройка почтового клиента}
	\label{fig:12}
\end{figure}

Для двух других узлов процесс аналогичен.

\subsection*{Настроить HTTP сервер, разместить там тестовую страницу с номером варианта, фамилией, номером группы, датой выполнения работы}

\begin{figure}[H]
	\begin{center}
		\includegraphics[scale=0.4]{img/13.png}
	\end{center}
	\caption{Настройка HTTP сервера}
	\label{fig:13}
\end{figure}

\begin{figure}[H]
	\begin{center}
		\includegraphics[scale=0.4]{img/14.png}
	\end{center}
	\caption{Настройка страницы}
	\label{fig:14}
\end{figure}

\subsection*{Проверить корректное прохождение сигнала между всеми узлами сети, доступность настроенных сервисов со стороны клиентов на ПК}

\begin{figure}[H]
	\begin{center}
		\includegraphics[scale=0.45]{img/15.png}
	\end{center}
	\caption{Проверка HTTP сервера}
	\label{fig:15}
\end{figure}

\begin{figure}[H]
	\begin{center}
		\includegraphics[scale=0.45]{img/16.png}
	\end{center}
	\caption{Проверка SMTP сервера}
	\label{fig:16}
\end{figure}

\begin{figure}[H]
	\begin{center}
		\includegraphics[scale=0.6]{img/17.png}
	\end{center}
	\caption{Проверка доступности хостов}
	\label{fig:17}
\end{figure}

\subsection*{Отметить широковещательные домены и домены коллизий на схеме}

\begin{figure}[H]
	\begin{center}
		\includegraphics[scale=0.5]{img/18.png}
	\end{center}
	\caption{Широковещательные домены расположены внутри прямоугольников, домены колизий - внутри кругов}
	\label{fig:18}
\end{figure}


\bibliographystyle{utf8gost705u}
\bibliography{51-biblio}
	
\end{document}
